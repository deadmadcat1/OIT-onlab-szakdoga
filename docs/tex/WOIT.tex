\subsection{Wavelet OIT}
\paragraph{}A \ac{WOIT}~\cite{aizenshtein2022wavelettransparency} a legújabban publikált technika a bemutatottak közül, amely épít többek között a \pageref{sec:MBOIT}. oldalon tárgyalt \ac{MBOIT}-ra is.
\paragraph{}Lényeges különbségek:
\begin{itemize}
	\item Ennek a technikának nem célja a fizikai szimulációt megközelítő pontosság, ellenkezőleg éppen az emberi észlelés sajátosságaira hagyatkozik, hogy végül egy ``hihetőbb'' képet alkosson hatékonyan, az átlátszóság bármilyen alkalmazásában, legyen az köd vagy haj.
	\item A láthatósági függvényt momentumok helyett wavelet\index{wavelet}-ekkel \cite{Mallat2008Wavelets} tárolja, amelyek magukban egyesítik a térbeli megoldások lokalitását és a frekvenciatartomány alapú megoldások tömöríthetőségét, így pontosabban megközelítve a valódi láthatósági függvényt, kisebb memóriaigénnyel.
\end{itemize}

\paragraph{}Ismét az elnyelési függvényt\index{elnyelési függvény} használjuk a számításokhoz.
\begin{gather*}
	A \approx \chi_{0,0}\phi_{0,0}\left(z\right)+\sum_{n,k}X_{n,k}\Psi_{n,k}\left(z\right)\\
	\chi_{0,0} = \sum\alpha_i\left(1-x_i\right)\\
	X_{n,k} = - \sum\alpha_i\Psi_{n,k}(x_i)
\end{gather*}

\paragraph{}Ahol $\Psi_{n,k}$ a wavelet maga.
$$\Psi_{n,k}\left(x\right) = 2^{-\frac{n}{2}}\cdot\left\lbrace\begin{matrix}
	2^nx-k	& 0\leq2^nx-k<0.5\\
	1+k-2^nx	& 0.5\leq2^x-k\leq1\\
	0			& \textrm{Máshol}
\end{matrix}\right\rbrace$$

