\subsection{Moment-Based OIT}
\label{sec:MBOIT}
\paragraph{}A \ac{MBOIT}~\cite{10.1145/3203206} egy relatíve új technika, amely a képtér pixelenkénti láthatósági függvényére\index{transmittance function}\index{láthatósági függvény} ad megközelítést vagy hatványos, vagy akár trigonometrikus momentumok használatával.

\paragraph{}A művelet két menetet követel meg, először a jelenet láthatósági függvényeinek meghatározásához, másodszor az átlátszó felületek összeállításához. Az átlátszatlan objektumok előre egy külön framebufferbe\index{framebuffer} lerajzolandók.

\paragraph{}A számításokat gazdaságosabb az elnyelési függvényre~(\autoref{eq:logtransmit})\index{elnyelési függvény} alapozni a logaritmus miatt.

\begin{gather}
	\label{eq:pointtransmittance} T(z_f) := \prod_{\substack{l=0\\z_l<z_f}}^{n-1}(1-\alpha_l)\\
	\label{eq:logtransmit}A(z_f) := -\ln\,T(z_f) = \sum_{\substack{l=0\\z_l<z_f}}^{n-1}-\ln(1-\alpha_l)
\end{gather}

\paragraph{}\Aref{eq:pointtransmittance}\,-\,es képlet a $z_f$ mélységbeli láthatósági függvénye a vizsgált pixelnek.
\Aref{eq:logtransmit} -es képlet pedig ugyanezen láthatósági függvény logaritmizálva, az elnyelési függvény $z_f$ mélységben.

Mivel $A(z_f)$ monoton a teljes mélységtartományon, képezhető belőle egy halmozott eloszlásfüggvény:
$$Z := \sum_{l=0}^{n-1}-\ln(1-\alpha_l)\cdot\delta_{z_l}$$
ahol $\delta_{z_l}$ egy Dirac-$\delta$ minden felület $z_l$ mélységében. A számított értékek tárolásához szükség van még egy momentum-generáló függvényre: $\boldsymbol{b} : [-1,1] \rightarrow \mathbb{R}^{m+1}$. Így a következő képlettel a vizsgált pixelbeli teljes láthatósági függvény hatékonyan eltárolható.

$$b:=\mathcal{E}_Z(\boldsymbol{b}):=\sum_{l=0}^{n-1}-\ln(1-\alpha_l)\cdot\boldsymbol{b}(z_l)$$

Árnyaláskor a láthatósági függvény\index{láthatósági függvény} visszanyerhető az alábbi módon:
$$T(z_f,\boldsymbol{b},\beta) = exp(-A(z_f,\boldsymbol{b},\beta))$$